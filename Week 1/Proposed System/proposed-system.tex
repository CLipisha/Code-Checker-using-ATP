\documentclass[]{article}

\begin{document}
	\title{\huge\textbf{Code-Checker using ATP}}
	\author{Niharika J, Lipisha C, Priti G  \\Guide: Kavita Wale \\ \\ \LARGE Fr. C. Rodrigues Institute of Technology, Vashi}
	\maketitle
	\pagebreak

\section{Proposed System}\label{proposed-system}

Code-Checker using Automated Theorem Proving is used for verifying the
logical correctness of a C program. It involves formulating an error
problem to be verified using a suitable logic. Theorem provers provide a
rigorous and reliable approach to proving of correctness, enabling and
establishing correctness of properties over infinite domains, properties
about complex data structures, and properties of recursive
structures etc. \\ \\
In this system , the user will give his syntactically-correct program as
an input to the Code Checker system. The system will check the
correctness of the program by comparing its rules which are inbuilt in
the system. It will then take appropriate action by displaying the
correct output in the form of message.

\subsection{Modules}\label{modules}

\begin{itemize}
 \item \textbf{User Interface} \\

The user will submit the program to be verified. He/she can verify the
correctness of their code and its logic by giving a source code as an
input to the system. Code-Checker using ATP accepts C programs to be
verified. \\

\item \textbf{Verification Module} \\

The system will take the C program from the user as input and it will
check for logical errors present in the program. There errors include
unreachable code blocks, infinite loops, infinite recursions, return
value mismatch etc. \\

\item \textbf{Notification Module} \\

In this module, after verifying the program, the system will inform the
user whether the C source code is logically correct or not. The
interaction with the user is provided in the form of messages. If the
program is verified correctly then a ``Successfully verified'' message
will be displayed. Else, the system will give warnings, which are
presented to the user in the form of error messages, according to which
the user may modify their program. For beginners in coding, this will
help them learn in an iterative fashion.

\end{itemize}

\end{document}