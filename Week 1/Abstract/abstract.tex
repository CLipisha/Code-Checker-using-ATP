\documentclass[]{article}

\begin{document}
	\title{\huge\textbf{Code-Checker using ATP}}
	\author{Niharika J, Lipisha C, Priti G  \\Guide: Kavita Wale \\ \\ \LARGE Fr. C. Rodrigues Institute of Technology, Vashi}
	\maketitle
	\pagebreak

\section{Abstract}\label{abstract}

The most common setback faced by beginners in programming is the need to
check whether the code will run robustly, given any form of input. Even
software written by professionals needs to undergo stress testing to
ensure its smooth field performance. This project, therefore, aims to
provide a mechanism to verify the given code using Automatic Theorem
Prover(ATP). \\

The user, i.e.~a programmer, can give his/her code as an input. This
input code(source code) is assumed to be syntactically correct. With
this, the Code Checker will determine the inputs being taken by the
source code. Once this information is gathered, the code is checked
again for common logical errors. For example, if there are any blocks of
code that are unreachable, Code Checker will detect it and display a
warning notification to the user. The user can then modify his/her
program accordingly. \\

Apart from the above example, Code Checker can identify other common
errors such as infinite loop conditions, possible infinite recursions,
return value mismatch etc. Other than the precise notification, nothing
else is printed on the user interface, keeping it clean and
clutter-free. Also, in this manner, a beginner coder can learn in an
iterative fashion.

\end{document}
