\documentclass[]{article}

\begin{document}
 
	\title{\huge\textbf{Code-Checker using ATP}}
	\author{Niharika J, Lipisha C, Priti G  \\Guide: Kavita Wale \\ \\ \LARGE Fr. C. Rodrigues Institute of Technology, Vashi}
	\maketitle
	\pagebreak


\section{Introduction}\label{introduction}

\subsection{Software Verification:}\label{software-verification}

Software verification is a discipline of software engineering whose goal
is to assure that software fully satisfies all the expected
requirements. In the context of hardware and software systems, software
verification is the act of proving or disproving the correctness of
intended algorithms underlying a system with respect to a certain formal
software specification or property, using rules. Software verification
can be helpful in proving the correctness of systems such as:
cryptographic protocols, combinational circuits, digital circuits with
internal memory, and software expressed as source code. The verification
of these systems is done by providing a formal proof on an abstract
model of the system, the correspondence between the mathematical model
and the nature of the system being otherwise known by construction.
Verifying the correctness of a program involves formulating a property
to be verified using a suitable logic such as first order logic or
temporal logic

There are two fundamental approaches to Software verification:

\begin{enumerate}
\def\labelenumi{\arabic{enumi}.}
\item
  \emph{\textbf{Dynamic Program Analysis}}, also known as Test or
  Experimentation - This is good for finding bugs.
\item
  \emph{\textbf{Static Program Analysis}}, also known as Analysis - This
  is useful for proving correctness of a program although it may result
  in false positives.
\end{enumerate}

In this system we will be implementing the software using static program
analysis.

\begin{itemize}

\item
  \textbf{Static Program Analysis:}
\end{itemize}

Static verification is the process of checking that software meets
requirements by inspecting the code before it runs. The analysis
verification method applies to verification by investigation,
mathematical calculations, logical evaluation, and calculations using
classical textbook methods or accepted general use computer methods.
Analysis includes sampling and correlating measured data and observed
test results with calculated expected values to establish conformance
with requirements.

Some of the implementation techniques of formal static analysis include:

\begin{enumerate}
\def\labelenumi{\arabic{enumi}.}
\item
  \emph{\textbf{Model checking}}, considers systems that have finite
  state or may be reduced to finite state by abstraction.
\item
  \emph{\textbf{Data-flow analysis}}, a lattice-based technique for
  gathering information about the possible set of values.
\item
  \emph{\textbf{Abstract interpretation}}, to model the effect that
  every statement has on the state of an abstract machine (i.e., it
  `executes' the software based on the mathematical properties of each
  statement and declaration). This abstract machine over-approximates
  the behaviours of the system: the abstract system is thus made simpler
  to analyse, at the expense of incompleteness (not every property true
  of the original system is true of the abstract system). If properly
  done, though, abstract interpretation is sound (every property true of
  the abstract system can be mapped to a true property of the original
  system). The Frama-c value analysis plugin and Polyspace heavily rely
  on abstract interpretation.
\item
  \emph{\textbf{Hoare logic}}, a formal system with a set of logical
  rules for reasoning rigorously about the correctness of computer
  programs. There is tool support for some programming languages (e.g.,
  the SPARK programming language (a subset of Ada) and the Java
  Modelling Language --- JML --- using ESC/Java and ESC/Java2, Frama-c
  WP (weakest precondition) plugin for the C language extended with ACSL
  (ANSI/ISO C Specification Language)).
\item
  \emph{\textbf{Symbolic execution}}, as used to derive mathematical
  expressions representing the value of mutated variables at particular
  points in the code.
\end{enumerate}

Examples of Static Program Verification: * Code conventions verification

\begin{itemize}
\item
  Bad practices (anti-pattern) detection
\item
  Software metrics calculation
\item
  Formal verification
\end{itemize}

\subsection{Implementation of Code Checker and
Deliverables:}\label{implementation-of-code-checker-and-deliverables}

Programmers while coding are unsure whether their code will run
efficiently, given any number of test cases (input). Software
verification helps these programmers to check the correctness of their
code and help them improve it. In this system the user (programmer) will
submit their code to be verified as an input to the system. This input
source code is assumed to be syntactically correct. Hence, it will be
checking for logical errors of the source code given as the input.

The Code Checker will be implemented in Python as the core language. The
input i.e.~the source code provided by the user will be in C language.

Once the source code is submitted as the input, the Code Checker will
check whether the code is logically correct or not. If the code is
logically sound (correct) the system will notify the user with a ``No
error'' (or a successful message) or with an error message that the
logic of his/her code is not correct corresponding to the test cases
applied to the code for checking the correctness.


\end{document}
